\documentclass[14pt,pdf,hyperref={unicode}]{beamer}

\usepackage[X2, T2A]{fontenc}
\usepackage[utf8x]{inputenc}
\usepackage[english,russian]{babel}

% Шрифт Times по умолчанию
\usepackage{pscyr}
\renewcommand{\rmdefault}{ftm}

% Многостраничные таблицы
\usepackage{tabularx,longtable}

% Пакет для листингов прогармм
\usepackage{minted}
\setminted{fontsize=\small, breaklines, autogobble, baselinestretch=1.1, tabsize=4, outencoding=utf8}

% Псевдокод
\usepackage{algorithm2e}
%\SetKwInput{KwData}{Исходные параметры}
%\SetKwInput{KwResult}{Результат}
%\SetKwInput{KwIn}{Входные данные}
%\SetKwInput{KwOut}{Выходные данные}
\SetKwIF{If}{ElseIf}{Else}{если}{то}{иначе если}{иначе}{конец условия}
\SetKwFor{While}{пока}{выполнять}{конец цикла}
%\SetKw{KwTo}{от}
%\SetKw{KwRet}{возвратить}
%\SetKw{Return}{возвратить}
%\SetKwBlock{Begin}{начало блока}{конец блока}
%\SetKwSwitch{Switch}{Case}{Other}{Проверить значение}{и выполнить}{вариант}{в противном случае}{конец варианта}{конец проверки значений}
%\SetKwFor{For}{цикл}{выполнять}{конец цикла}
%\SetKwFor{ForEach}{для каждого}{выполнять}{конец цикла}
%\SetKwRepeat{Repeat}{повторять}{до тех пор, пока}
%\SetAlgorithmName{Алгоритм}{алгоритм}{Список алгоритмов}

% Библитека для рисования фигур
\usepackage{tikz}

% отключить клавиши навигации
\setbeamertemplate{navigation symbols}{}

% тема оформления
\usetheme{Singapore}

% цветовая схема
\usecolortheme{dove}


\title{Компилируемый язык описания программ для эмулятора машины Тьюринга}
\subtitle{Разработка и использование компилятора}
\author{Воробьев Л.О.} 
\date{\today} 
% \logo{\includegraphics[height=5mm]{images/logo.png}\vspace{-7pt}}

\begin{document}

% титульный слайд
\begin{frame}
\titlepage
\end{frame} 

\begin{frame}
\frametitle{Цель работы} 
\begin{itemize}
	\item Создание специализированного компилируемого языка описания программ для многоленточных и одноленточных машин Тьюринга.
	\item Разработка и реализация компилятора для созданного языка 
	\item Написание эффективных алгоритмов интерпретации скомпилированных программ.
\end{itemize}
\end{frame}

\begin{frame}{Актуальность}
	\begin{itemize}
		\item Объемность и сложность понимания программ для машины Тьюринга обуславливает высокую актуальность создания языка, упрощающего процесс написания и понимания таких программ.
		\item Новый компилируемый язык позволит увеличить производительность программного обеспечения, которое требует использования машины Тьюринга.
	\end{itemize}
\end{frame}

\begin{frame}{Математическая запись программ для машины Тьюринга}
	Программа для машины Тьюринга --- это конечный набор команд вида:
	\begin{equation}\label{eq:command}
	q_s a_i \rightarrow q_t a_j d
	\end{equation}
	для многоленточных машин Тьюринга:
	\begin{equation}\label{eq:command_2}
	q_s \left( a_{i1},\,a_{i2},\,\ldots a_{in} \right) \rightarrow q_t \left( a_{j1},\,a_{j2},\,\ldots a_{jn} \right) d
	\end{equation}
	где $q$ --- номера состояний; $a$ --- содержимое ячеек; $d$ --- направления движения лент.
\end{frame}

\begin{frame}{Требования к создаваемому языку}
	\begin{block}{Проблема}
		Повтор значений из-за отсутствия действия. Например:
		$$ q_1 1 \rightarrow q_1 1 L ; \quad q_1 2 \rightarrow q_2 2 E ; \quad q_2 3 \rightarrow q_2 2 E $$
	\end{block}
		
	\begin{block}{Решение}
		Избавится от избыточности путем разделения команды на несколько операторов.
	\end{block}
\end{frame}

\begin{frame}{Требования к создаваемому языку}
	\begin{block}{Проблема}
		Состояния с командами, выполняющими одинаковые действия. Например:
		$$ q_1 1 \rightarrow q_2 1 R ; \quad q_2 1 \rightarrow q_3 1 E ; \quad q_3 1 \rightarrow q_1 1 E $$
	\end{block}
	
	\begin{block}{Решение}
		Избавится от избыточности путем использования шаблонных состояний.
	\end{block}
\end{frame}

\begin{frame}{Требования к создаваемому языку}
	\begin{block}{Проблема}
		Одинаковые действия при разных условиях. Например:
		$$
		q_1 (1,1) \rightarrow q_2 (1,0) (R,E) ; \quad q_1 (2,1) \rightarrow q_2 (2,0) (R,E)
		$$
	\end{block}
	
	\begin{block}{Решение}
		Избавится от избыточности путем использования условных выражений.
	\end{block}
\end{frame}

\begin{frame}{Синтаксис разработанного языка}
	\begin{columns}
		\begin{column}{5cm}
			Пример программы
			\inputminted[tabsize=2]{text}{../../examples/increment.mtr}
		\end{column}
		
		\begin{column}{7cm}
			Форма БНФ (фрагмент)
			\inputminted[fontsize=\scriptsize,tabsize=2,firstline=50]{bnf}{../article/bnf/turing.bnf}
		\end{column}
	\end{columns}
\end{frame}

\begin{frame}{Шаблоны состояний}
	\begin{block}{Форма БНФ}
		\inputminted[fontsize=\scriptsize,tabsize=2,firstline=8,lastline=10]{bnf}{../article/bnf/turing.bnf}
	\end{block}
	\inputminted[tabsize=2]{text}{syntax/template.txt}
\end{frame}

\begin{frame}{Условные выражения}
	\begin{itemize}
		\item Простое выражение. Например: \texttt{'1'}; \texttt{'\textbackslash x0F'}; \texttt{14}; \texttt{nul}.
		\item Использование логических операторов \texttt{and}, \texttt{or}. Например: \texttt{'1' or nul}.
		\item Перечисление в круглых скобках. Например, \texttt{(«abc», '*')} соответствует \texttt{('a','*')or('c','*')or('b','*')}.
		\item Операторы сравнения \texttt{==,!=,<,>,<=,>=}. \texttt{(«abc», «abc») and \$2 == \$1}.
	\end{itemize}
\end{frame}

\begin{frame}{Работа компилятора}
	\begin{center}
		\includegraphics[scale=.45]{fig/compiler}
	\end{center}
\end{frame}

\begin{frame}{Использование компилятора mtrc}
	\begin{center}
		\begin{tikzpicture}[source/.style={draw, inner sep = 10pt},
		pre/.style={<-,shorten <=1pt,>=stealth,semithick},
		post/.style={->,shorten >=1pt,>=stealth,semithick}]
		\node[source] (a) at (0,0) {main.c};
		\node[source] (b) at (4,0) {increment.mtr};
		\node (c) at (0,-1.5) {cc}
			edge[pre] (a);
		\node (d) at (4,-1.5) {mtrc}
			edge[pre] (b);
		\node[source] (e) at (0,-3) {main.o}
			edge[pre] (c);
		\node[source] (f) at (4,-3) {increment.obj}
			edge[pre] (d);
		\node[source] (g) at (5,-4.5) {statestep.lib};
		\node (h) at (2, -4.5) {link}
			edge[pre] (e)
			edge[pre] (f)
			edge[pre] (g);
		\node[source] at (-1,-4.5) {program.exe}
			edge[pre] (h);
		\end{tikzpicture}
	\end{center}
\end{frame}

\begin{frame}{Устройство выходной программы компилятора mtrc}
\framesubtitle{increment.obj}
\begin{columns}
	\begin{column}[t]{3.7cm}
		Машинный код
		\inputminted[fontsize=\scriptsize]{text}{example/disasm.txt}
	\end{column}
	
	\begin{column}[t]{4.3cm}
		Ассемблер x86
		\inputminted[fontsize=\scriptsize]{asm}{example/increment.asm}
	\end{column}
	
	\begin{column}[t]{4cm}
		Язык C
		\inputminted[fontsize=\scriptsize]{c}{example/increment.c}
	\end{column}
\end{columns}
\end{frame}

\begin{frame}{Структура правила перехода}
	Выходная программа проверяет условия и возвращает 32-разрядные правила перехода:
	\begin{center}
		\begin{tikzpicture}[shorten >=1pt,>=stealth,
			flag name/.style={anchor=north west,inner sep=0pt}]
		\draw (0,0) rectangle (10,.75);
		\foreach \x in {2.5,2.8125,3.125,3.4375,3.75,5,7.5}
			\draw (\x,0) -- +(up:.75);
		\foreach \x/\t in {0/31,2.5/23,3.75/19,5/15,7.5/7,10/0}
			\node[anchor=south] at (\x,.75) {\t};
		\foreach \x/\t in {4.375/dir,6.25/a\_sec,8.75/a\_fst}
			\node[anchor=south] at (\x,0) {\t};
		\draw[] (3.59375,.25) -- (3.75,-.25) node[flag name] {a\_fst\_const};
		\draw[] (3.28125,.25) -- (3.75,-1) node[flag name] {a\_sec\_const};
		\draw[] (2.96875,.25) -- (3.75,-1.75) node[flag name] {f\_repeat};
		\draw[] (2.65625,.25) -- (3.75,-2.5) node[flag name] {f\_norule};
		\draw[thick,->] (1.25, -.15) -> +(down:.5);
		\draw (0,-1.5) rectangle +(2.5, .75);
		\draw (.625,-1.5) -- +(up:.75);
		\draw (0,-3) rectangle +(2.5, .75);
		\node[anchor=south] at (1.5625,-1.5) {q\_num};
		\node[anchor=south] at (1.25,-3) {tape\_num};
		\end{tikzpicture}
	\end{center}
\end{frame}

\begin{frame}{Алгоритм интерпретации полученных программ}
\framesubtitle{Библиотека statestep.lib}
\begin{center}
\scalebox{.8}{ 
	\begin{algorithm}[H]
		\SetAlgoLined
		\DontPrintSemicolon
		$q$ := $q_0$\;
		\While{$q$ != $q_z$}{
			читать $a_1$ и $a_2$ с лент\;
			получить правило перехода\;
			\uIf{установлен флаг f\_repeat}{
				запустить циклическое выполнение правил перехода\;
			}
			\uElseIf{установлен флаг f\_norule}{
				прервать работу с ошибкой\;
			}
			\Else{
				выполнить правило перехода\;
			}
		}
	\end{algorithm}
}
\end{center}
\end{frame}

\begin{frame}{Перспективы разработки}
	Планируется добавить следующие возможности:
	\begin{itemize}
		\item использование препроцессора;
		\item генерация машинного кода для разных платформ;
		\item поддержка архитектуры x64;
		\item оптимизация алгоритмов компилятором.
	\end{itemize}
\end{frame}

\begin{frame}{Заключение}
	\begin{itemize}
		\item Разработанный язык позволяет избавится от избыточности представления программ для машины Тьюринга.
		\item Применение разработанного языка позволит повысить быстродействие программного обеспечения, которое использует механизм машины Тьюринга: синтаксические анализаторы, системы шифрования, и т.д.
	\end{itemize}
\end{frame}

\begin{frame}[plain]
	\begin{center}
		Спасибо за внимание!
	\end{center}
\end{frame}

\end{document}